\subsection{Primal-Dual-Methods} \label{subsec:-primal-dual-methods}
We consider the linear programming problem in standard form:
\[
    \min c^T x, \text{ subject to } Ax = b, x \geq 0 \tag{14.1}\label{lp: 14.1}
\]
where $c$ and $x$ are vectors in $\mathcal{R}^n$, $b$ is a vector in $\mathcal{R}^m$, and A is an
$m\times n$ matrix with full row rank. (As in Chapter 13, Pre-processing the problem to remove dependent rows from A
is necessary).The dual problem for\ref{lp: 14.1} is
\[
    \max b^T \lambda, \text{ subject to } A^T \lambda + s = c, s\geq 0 \tag{14.2}\label{dual: 14.2}
\]
where $\lambda$ is a vector in $\mathcal{R}^m$, and $s$ is a vector in $\mathcal{R}^n$.Solutions of\eqref{lp: 14.1}
and\eqref{dual: 14.2} are characterized by the KKT conditions:

\begin{align}
    A^T\lambda + s &= c \tag{14.3a}\label{KKT: 14.3a}\\
    Ax & = b \tag{14.3b}\label{KKT: 14.3b}\\
    x_i s_i & = 0, \quad i = 1,2,\cdots, n \tag{14.3c} \label{KKT: 14.3c}\\
    (x, s) & \geq 0 \tag{14.3d}\label{KKT: 14.3d}
\end{align}

Primal-dual method find solutions $(x^*, \lambda^*, s^*)$ of this system by applying variants of Newton's method to the
three equalities in (14.3) and modifying the search directions and step lengths so that the inequalities
$(x, s) \geq 0$ are satisfied strictly at every iteration.The equations\eqref{KKT: 14.3a},\eqref{KKT: 14.3b}
,\eqref{KKT: 14.3c} are linear or only mildly nonlinear and so are not difficult to solve by themselves.However, the problem
becomes much more difficult when we add the non-negativity requirement\eqref{KKT: 14.3d}
\par To derive primal-dual interior-point methods we restate the optimality conditions (14.3) in a slightly different form
by means of a mapping $F$ from $\mathbb{R}^{2n + m}$ to $\mathbb{R}^{2n + m}$
\begin{align}
    F(x, \lambda, s) =
    \begin{bmatrix}
        A^T\lambda + s - c\\
        Ax - b \\
        XSe
    \end{bmatrix} = 0 \tag{14.4a}\label{KKT: 14.4a}\\
    (x, s) \tag{14.4b} \label{KKT:14.4b}\geq 0
\end{align}
where
\[
    X = diag(x_1, x_2,\cdots,x_n), \qquad S = diag(s_1, s_2, \cdots, s_n) \tag{14.5}\label{KKT: 14.5}
\]
and $e = (1,1,\cdots, 1)^T$.Primal-dual methods generate iterates $x^k, \lambda^k, s^k $ that satisfy the
bounds\eqref{KKT:14.4b} strictly, that is $x^k \geq 0$ and $s^k \geq 0$.This property is the origin of the term
interior-point.The primal-dual interior point methods have two basic ingredients: a procedure for determining the step
and a measure of desirability of each point in the search space.An important component of hte measure of desirability
is the average value of the pairwise products $x_i s_i ,i = 1,2,\cdots ,n $ which are all positive when $x > 0$ and $s > 0$.
This quantity is known as the duality measure and is defined as follows:
\[
    \mu = \frac{1}{n} \sum_{i=1}^n x_i s_i = \frac{x^T s}{n} \tag{14.6} \label{ipm: 14.6}
\]